\documentclass[11pt]{book}
\usepackage{fullpage}
\usepackage{longtable}
\usepackage{bytefield}
\usepackage{color}
\usepackage[table]{xcolor}
\usepackage[bookmarksopen=true]{hyperref}


\usepackage{sectsty}
\allsectionsfont{\usefont{OT1}{phv}{bc}{n}\selectfont}

\usepackage[Lenny]{fncychap}

\hypersetup{%
pdftitle={AjarDSP User's Guide},
pdfauthor={Markus Lavin <markusl.se78@gmail.com>},
pdfsubject={AjarDSP User's Guide},
pdfkeywords={AjarDSP}}

% Set up hyperlink colors
\definecolor{darkred}{rgb}{0.5,0,0}
\definecolor{darkgreen}{rgb}{0,0.3,0}
\definecolor{darkblue}{rgb}{0,0,0.5}
\definecolor{darkbrown}{rgb}{0.28,0.07,0.07}
\hypersetup{%
  colorlinks=true,
  citecolor=darkblue,
  urlcolor=darkgreen,
  linkcolor=darkred,
  menucolor=darkbrown}

\renewcommand{\familydefault}{\sfdefault}
\newcommand{\HRule}{\rule{\linewidth}{0.5mm}}

\begin{document}

\begin{titlepage}
\begin{center}
\HRule \\[0.4cm]
{\Huge \textbf{AjarDSP User's Guide}}\\[0.8cm]
\HRule \\[1.5cm]
\emph{URL:}\\
\texttt{http://code.google.com/p/ajardsp/}\\
\vfill
\begin{flushleft}
\begin{minipage}{0.4\textwidth}
\begin{flushleft} \large
\emph{Author:}\\
Markus Lavin \texttt{markusl.se78@gmail.com}
\end{flushleft}
\end{minipage}
\end{flushleft}
\end{center}
\end{titlepage}

%%%%%%%%%%%%%%%%%%%%%%%%%%%%%%%%%%%%%%%%%%%%%%%%%%%%%%%%%%%%%%%%%%%%%%%%
\chapter{Architecture}

\section{Registers}
This section describes the registers available in AjarDSP
architecture. There are two main register files (accrf and ptrrf) and
group of lously connected special purpose registers spread out thorugh
the design called the special registers.
\subsection{Accumulator Register File (ACCRF)}
The accumulator register file (accrf) consits of eight 40-bit
accumulator registers. Each such accumualtor register is divided into
three parts; guard bits (typically 8-bits but configurable), a 16-bit
high part and a 16-bit low part.  \break

\begin{bytefield}{40}
\bitheader[b]{0,15,16,31,32,39} \\
\bitbox{8}{accNg} &
\bitbox{16}{accNh} &
\bitbox{16}{accNl} \\
\end{bytefield}

The high and low parts (accNh and accNl) may be addressed individually
by some instructions (those operating on 16-bit data) but the guard
bits are not directly addressable. Access to the guard bits must be
done via the BMU by means of downshifting.

\subsection{Pointer Register File (PTRRF)}
The pointer register file (ptrrf) consists of eight 16-bit
registers. These registers are only accessable from the load store
units (LSU) and their purpose is to addressing memory.  \break

\subsection{Special registers}
As indicated above the special registers are not located togehter in a
register file but instead spread out through the design (e.g. retpc is
near the PCU and sp is near the LSUs).
\break
The architecture provides the follwing special purpose registers:
\begin{center}
  \begin{tabular}{ | l | l |}
    \hline
    \cellcolor{lightgray} Register & \cellcolor{lightgray} Description \\ \hline
    \textbf{sp} & Stack Pointer \\ \hline
    \textbf{retpc} & Return PC (link register) \\ \hline
    \textbf{satctrl} & Saturation control \\ \hline
    \textbf{mulsign} & Multiplication operand sign \\ \hline
    \textbf{masksel} & Pointer mask selector \\ \hline
    \textbf{mask0}   & Pointer mask register 0 \\ \hline
    \textbf{mask1}   & Pointer mask register 1 \\ \hline
    \textbf{modsel}  & Pointer modification selector \\ \hline
    \textbf{mod0}    & Pointer modification register 0 \\ \hline
    \textbf{mod1}    & Pointer modification register 1 \\ \hline
    \textbf{bitrev}  & Reverse carry pointer modification enable register \\ \hline
    \textbf{gpio}    & General Purpose Input Output register \\
    \hline
  \end{tabular}
\end{center}



\begin{bytefield}{16}
\bitheader[b]{0,15} \\
\bitbox{16}{ptrN} \\
\end{bytefield}

%%%%%%%%%%%%%%%%%%%%%%%%%%%%%%%%%%%%%%%%%%%%%%%%%%%%%%%%%%%%%%%%%%%%%%%%
\section{Program Control Unit - PCU}
The PCU is responsible for program flow control and instruction fetching.

%%%%%%%%%%%%%%%%%%%%%%%%%%%%%%%%%%%%%%%%%%%%%%%%%%%%%%%%%%%%%%%%%%%%%%%%
\section{Load Store Unit - LSU}

\subsection{Addressing modes}
The memory is 16-bit word addressed (meaning that given address $A$
then address $A+1$ is the address of the next 16-bit word in
memory). Data layout is big-endian??

The architecture supports three different addressing modes. The
current addressing modes affects the operation of the load/store
instructions with automatic pointer update (i.e. ldinc16/32 and
stinc16/32 instructions).

\begin{description}
  \item[linear mode] This is the normal adderssing mode where pointers
    will be updated in a linear fashion.
  \item[circular mode] In this addressing mode a mask register is used
    to mask the updated pointer and effectively making the pointer
    wrap once its value goes outside the mask.
  \item[circular mode with reverse carry update] This is similar to
    circular mode but the modification value is choosen in such a way
    that together with the bit-reversed update the address bits inside
    the circular buffer will updated with reversed order.
\end{description}



%%%%%%%%%%%%%%%%%%%%%%%%%%%%%%%%%%%%%%%%%%%%%%%%%%%%%%%%%%%%%%%%%%%%%%%%
\section{Computational Unit - CU}

%%%%%%%%%%%%%%%%%%%%%%%%%%%%%%%%%%%%%%%%%%%%%%%%%%%%%%%%%%%%%%%%%%%%%%%%
\section{Bit Manipulation Unit - BMU}

%%%%%%%%%%%%%%%%%%%%%%%%%%%%%%%%%%%%%%%%%%%%%%%%%%%%%%%%%%%%%%%%%%%%%%%%
\section{Pipeline issues}

%%%%%%%%%%%%%%%%%%%%%%%%%%%%%%%%%%%%%%%%%%%%%%%%%%%%%%%%%%%%%%%%%%%%%%%%
\newpage
\chapter{Instruction set}
\newpage
\section{Bit Manipulation Unit (BMU)}
\input{bmu-insns.tex}
\newpage
\section{Computational Unit (CU)}
\input{cu-insns.tex}
\newpage
\section{Load Store Unit (LSU)}
\input{lsu-insns.tex}

\end{document}
